\include{settings}

\begin{document}	% начало документа

% Титульная страница
\include{titlepage}

% Содержание
\include{ToC}


\section{Цель работы}
Изучить воздействие фильра нижних частот на тестовый синал с шумом.

\section{Постановка задачи}
Сгенерировать тестовый гармонический сигнал с шумом, синтезировать ФНЧ, отфильтровать сигнал с шумом. Посмотреть, как ФНЧ влияет на спектр сигнала.

\section{Теоретическая информация}
\subsection{Генерация гармонического сигнала с шумом}
Для генерации гармонического сигнала воспользуемся формулой $s(t) = A*cos(2*\pi * f*t + \varphi)$, где А - амплитуда сигнала, f - частота, t - вектор отсчетов времени, $\varphi$ - смещение по фазе.

Добавление шума представляет собой суммирование исходного сигнала с сигналом другой частоты, как правило(но не всегда), меньшей амплитуды.

\subsection{Фильтр нижних частот}
Фильтрация представляет собой операцию поточечного домножения сигнала в частотной области на некоторую точечную последовательность коэффициентов, определяемую частотой исходного сигнала. Например, ФНЧ реализуется домножением всех частот, больших некоторой частоты $f_0$, называемой частотой среза, на число 0. Реально же, последовательность коэффициентов нелинейна, из-за чего происходит не обнуление, а ослабление сигналов выбранных частот. Таким образом, фильтр может ослаблять сигналы на одной частоте, и оставлять неизменными, или даже усиливать, на другой частоте. АЧХ такого фильтра представлена на Рис.\ref{pic:filter_theor}:
\begin{figure}[H]
	\begin{center}
		\includegraphics[scale=0.7]{filter_theor}
		\caption{АЧХ фильра нижних частот} 
		\label{pic:filter_theor} % название для ссылок внутри кода
	\end{center}
\end{figure}

Представляется крайне трудным смоделировать идеальную характеристику фильтра, поэтому используют удобные и простые в реализации аппроксимации - Чебышева, Баттерворта, Бесселя, эллиптические и т.д. Используем фильтр Баттерворта 4 порядка. АЧХ фильтра Баттерворта порадка n можно вычислить по формуле:
\begin{equation}
	G^2(\omega) = \frac{G^2_0}{1 + \big( \frac{\omega}{\omega _c} \big) ^{2n}}
\end{equation}
где n - порядок фильтра, $\omega _c$ - частота среза, $G_0$ - коэффициент усиления на нулевой частоте.

\section{Ход работы}
Код программы представлен ниже \ref{code:code}:
\lstinputlisting[
	label=code:code,
	caption={Код в МатЛаб},% для печати символ '_' требует выходной символ '\'
]{Code.m}
Его можно условно разделить на 2 части - генерация сигнала и выведение его графика и его спектра на экран, и фильтрация этого сигнала, с последующим выведением тех же графиков

\subsection{Генерация гармонического сигнала с шумом}
Для начала получим обычный гармонический сигнал. Пусть его частота будет 20 Гц. Сгенерированный сигнал представлен на рисунке \ref{pic:signal}:
\begin{figure}[H]
	\begin{center}
		\includegraphics[scale=0.7]{signal}
		\caption{Гармонический сигнал $s(t) = A*cos(2*\pi * f*t + \varphi)$} 
		\label{pic:signal} % название для ссылок внутри кода
	\end{center}
\end{figure}
На графике видимо обычную синусоиду.

Затем сгенерироем еще одну синусоиду с другой, более высокой частотой, прибавив его к уже полеченной гармонике. Результат внесния шума в сигнал виден на рисунке \ref{pic:signal2}:
\begin{figure}[H]
	\begin{center}
		\includegraphics[scale=0.7]{signal_noise}
		\caption{Гармонический сигнал с шумом} 
		\label{pic:signal_noise} % название для ссылок внутри кода
	\end{center}
\end{figure}

Далее получим спектр сигнала с помощью преобразования Фурье, встроенного в МатЛаб. Спектр гармонического сигнала с шумом приведен на рисунке \ref{pic:signal_noise_spectrum}:
\begin{figure}[H]
	\begin{center}
		\includegraphics[scale=0.7]{signal_noise_spectrum}
		\caption{Спектр зашумленной гармоники} 
		\label{pic:signal_noise_spectrum} % название для ссылок внутри кода
	\end{center}
\end{figure}
В сигнале присутствуют несколько гармоник разной частоты.

\subsection{Фильтрация сигнала}
Для фильтрации будем использовать ФНЧ Баттерворта 4-ого порядка. Коэффициенты фильтра получим с помощью встроенной в МатЛаб функции butter. В качестве аргумента указываем порядок фильтра и величину $Fn*2/Fd$, где Fn - частота полезного сигнала, а Fd - частота дискретизации.
Полученные коэффициенты задаются как аргументы в функции фильтрации filter. На выходе будет иметься отфильтрованый сигнал. Его можно видеть на рисунке \ref{pic:filter_signal}:
\begin{figure}[H]
	\begin{center}
		\includegraphics[scale=0.7]{filter_signal}
		\caption{Сигнал после прохождения фильтра} 
		\label{pic:filter_signal} % название для ссылок внутри кода
	\end{center}
\end{figure}
На графике представлен отфильтрованый сигнал. Амплитуда уменьшена за счет коэффициента ослабления фильтра, так же сигнал устанавливается с задержкой.

Спектр данного сигнала, полученный также с помощью преобразования Фурье, приведен на рис. \ref{pic:filter_signal_spectrum}:
\begin{figure}[H]
	\begin{center}
		\includegraphics[scale=0.7]{filter_signal_spectrum}
		\caption{Спектр отфильтрованного сигнала} 
		\label{pic:filter_signal_spectrum} % название для ссылок внутри кода
	\end{center}
\end{figure}
На рисунке видна одна гармоника, т.е. фильтр отсек гармонику шума, внесенного нами в сигнал.
Ниже представлена модель Simulink КИХ-фильтра и сигнал до и после фильрации.
\begin{figure}[H]
	\begin{center}
		\includegraphics[scale=0.7]{Model}
		\caption{Модель Simulink} 
		\label{pic:Model} % название для ссылок внутри кода
	\end{center}
\end{figure}
\begin{figure}[H]
	\begin{center}
		\includegraphics[scale=0.7]{Model_filt}
		\caption{Сигнал до (желтый) и после (синий) КИХ-фильтра} 
		\label{pic:Model_filt} % название для ссылок внутри кода
	\end{center}
\end{figure}
Сигнал формируется с задержкой, в отличие от случая фильтра Баттерворта (с БИХ)

\section{Выводы}

Нами исследовано прохождение сигнала через фильтр нижних частот. На примере зашумленного гармонического сигнала удалось получить представление о том, что при фильтрации частотная характеристика сигнала поточечно домножается на желаемую картину фильтрации (АЧХ). Сложность состоит в том, что получить в качестве окна идеальный прямоугольник технически невозможно. Поэтому используются различные методы аппроксимации АЧХ фильтра. Неидеальностью АЧХ фильтра можно объяснипть неполное подавление шума, особенно на частотах, близких к частоте среза, т.к. аппроксимация иммеет неидеальный наклон кривой после частоты среза.

\end{document}
